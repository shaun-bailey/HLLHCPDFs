\documentclass[11pt,a4paper]{article}
\usepackage{placeins}
\usepackage{graphicx}
\usepackage{xcolor}
\usepackage{float}
\usepackage{afterpage}
\usepackage{amssymb,amsmath}
\usepackage{multirow,booktabs,multirow}
\usepackage{cite}
\usepackage[colorlinks=true, linkcolor=black!50!blue, urlcolor=blue, citecolor=blue, anchorcolor=blue]{hyperref}
\usepackage[font=small,labelfont=bf,margin=0mm,labelsep=period,tableposition=top]{caption}
\usepackage[a4paper,top=3cm,bottom=2.5cm,left=2.5cm,right=2.5cm,bindingoffset=0mm]{geometry}
\setlength{\unitlength}{1mm}

\usepackage{tabularx}
\newcolumntype{C}[1]{>{\centering\arraybackslash}p{#1}}

%%%%%%%%%%%%%%%%%%%%%%%%%%%%%%%%%%%%%%%%%%%%%%%%%%%%%%%%%%%%%

\def\smallfrac#1#2{\hbox{$\frac{#1}{#2}$}}
\newcommand{\be}{\begin{equation}}
\newcommand{\ee}{\end{equation}}
\newcommand{\bea}{\begin{eqnarray}}
\newcommand{\eea}{\end{eqnarray}}
\newcommand{\bi}{\begin{itemize}}
\newcommand{\ei}{\end{itemize}}
\newcommand{\ben}{\begin{enumerate}}
\newcommand{\een}{\end{enumerate}}
\newcommand{\la}{\left\langle}
\newcommand{\ra}{\right\rangle}
\newcommand{\lc}{\left[}
\newcommand{\rc}{\right]}
\newcommand{\lp}{\left(}
\newcommand{\rp}{\right)}
\newcommand{\as}{\alpha_s}
\newcommand{\aq}{\alpha_s\left( Q^2 \right)}
\newcommand{\amz}{\alpha_s\left( M_Z^2 \right)}
\newcommand{\aqq}{\alpha_s \left( Q^2_0 \right)}
\newcommand{\aqz}{\alpha_s \left( Q^2_0 \right)}
\newcommand{\Ord}{\mathcal{O}}
\newcommand{\MSbar}{\overline{\text{MS}}}
\def\toinf#1{\mathrel{\mathop{\sim}\limits_{\scriptscriptstyle
{#1\rightarrow\infty }}}}
\def\tozero#1{\mathrel{\mathop{\sim}\limits_{\scriptscriptstyle
{#1\rightarrow0 }}}}
\def\toone#1{\mathrel{\mathop{\sim}\limits_{\scriptscriptstyle
{#1\rightarrow1 }}}}
\def\frac#1#2{{{#1}\over {#2}}}
\def\gsim{\gtrsim}
\def\lsim{\lesssim}    
\newcommand{\mrexp}{\mathrm{exp}}
\newcommand{\dat}{\mathrm{dat}}
\newcommand{\one}{\mathrm{(1)}}
\newcommand{\two}{\mathrm{(2)}}
\newcommand{\art}{\mathrm{art}} 
\newcommand{\rep}{\mathrm{rep}}
\newcommand{\net}{\mathrm{net}}
\newcommand{\stopp}{\mathrm{stop}}
\newcommand{\sys}{\mathrm{sys}}
\newcommand{\stat}{\mathrm{stat}}
\newcommand{\diag}{\mathrm{diag}}
\newcommand{\pdf}{\mathrm{pdf}}
\newcommand{\tot}{\mathrm{tot}}
\newcommand{\minn}{\mathrm{min}}
\newcommand{\mut}{\mathrm{mut}}
\newcommand{\partt}{\mathrm{part}}
\newcommand{\dof}{\mathrm{dof}}
\newcommand{\NS}{\mathrm{NS}}
\newcommand{\cov}{\mathrm{cov}}
\newcommand{\gen}{\mathrm{gen}}
\newcommand{\cut}{\mathrm{cut}}
\newcommand{\parr}{\mathrm{par}}
\newcommand{\val}{\mathrm{val}}
\newcommand{\tr}{\mathrm{tr}}
\newcommand{\checkk}{\mathrm{check}}
\newcommand{\reff}{\mathrm{ref}}
\newcommand{\Mll}{M_{ll}}
\newcommand{\extra}{\mathrm{extra}}
\newcommand{\draft}[1]{}
\newcommand{\comment}[1]{{\bf \it  #1}}
\newcommand{\muf}{\mu_\text{F}}
\newcommand{\mur}{\mu_\text{R}}

\def\beq{\begin{equation}}  
\def\eeq{\end{equation}}  


\def\({\left(}
\def\){\right)}
\def\[{\left[}
\def\]{\right]}
\let\originalleft\left
\let\originalright\right
\renewcommand{\left}{\mathopen{}\mathclose\bgroup\originalleft}
\renewcommand{\right}{\aftergroup\egroup\originalright}


\numberwithin{equation}{section}
\numberwithin{figure}{section}
\numberwithin{table}{section}

%\let\sectionold\section
%\renewcommand\section[2][]{%
%\sectionold{\boldmath #2}}

\let\oldsubsection\subsection
\renewcommand\subsection[2][\subsectiontoc]{%
  \def\subsectiontoc{#2}%
  \oldsubsection[#1]{\boldmath #2}%
}

\let\oldsubsubsection\subsubsection
\renewcommand\subsubsection[2][\subsubsectiontoc]{%
  \def\subsubsectiontoc{#2}%
  \oldsubsubsection[#1]{\boldmath #2}%
}


\newcommand{\tmop}[1]{\ensuremath{\operatorname{#1}}}
\newcommand{\tmtextit}[1]{{\itshape{#1}}}
\newcommand{\tmtextrm}[1]{{\rmfamily{#1}}}
\newcommand{\tmtexttt}[1]{{\ttfamily{#1}}}


\usepackage{xcolor}
%\usepackage{showlabels}
\bibliographystyle{JHEP}

\begin{document}
\newgeometry{top=1.5cm,bottom=1.5cm,left=2.5cm,right=2.5cm,bindingoffset=0mm}
\begin{titlepage}
\thispagestyle{empty}
\noindent

\begin{center}
  {\LARGE \bf\boldmath
Ultimate Parton Distributions\\[0.3cm] at the High-Luminosity LHC}
\vspace{1.3cm}


Lucian, Jun, Juan, Shaun, Rabah, ....


\vspace{1.0cm}

{\bf \large Abstract}

\end{center}

Notes of the ultimate precision that can be expected
on parton distributions at the high-luminosity LHC.


\vspace{0.8cm}

\end{titlepage}

\restoregeometry

\tableofcontents

\section{Introduction}
\label{sec:introduction}

The high-luminosity (HL) upgrade of the Large Hadron Collider (LHC)
will operate between 2026 and 2037, extending by more than ten years
the LHC lifetime as compared to Runs II and III.
%
Thanks to extensive improvements both in the accelerator chain and
in the detectors, the HL-LHC aims to accumulate a total
integrated luminosity of around $\mathcal{L}=3$ ab$^{-1}$.
%
This large luminosity will make possible a number of important
physics analysis from the measurement of the couplings
of the Higgs boson to the second generation quarks and leptons
to that of its self-coupling.

In order to fully exploit the physics potential of the HL-LHC,
it is of paramount importance to be able to reduce as much as possible]the
  various sources of uncertainties affecting the theoretical calculations.
  %
  One of the most relevant ones is the theoretical uncertainty
  associated to our imperfect knowledge of the quark
  and gluon structure of the proton, encapsulated
  by the parton distribution functions.


\section{Settings}
\label{sec:settings}

In this section we present the settings of the exercise that
we are going to carry out.
%
First of all we discuss the pseudo-data that will be used as an input to
the PDF fit.
%
Then we explain how the corresponding theoretical predictions have been produced.

\subsection{Pseudo-data generation}

Some of the processes that we will consider in this exercise
are:

\begin{itemize}

\item Drell-Yan - low to high mass, central to forward,
  relevant for quark flavour separation.

\item   Differential top quark pair production,
  relevant for the large $x$ gluon.

\item Inclusive jet production,
  useful to constrain both quarks and gluons.

  Measurements of inclusive jet and dijet production from ATLAS at 13 TeV are
  already available~\cite{Aaboud:2017wsi}
  based on a small luminosity of $\mathcal{L}=3.2$ fb$^{-1}$.
  %
  We would need to extrapolate these measurements to $\mathcal{L}=3$ ab$^{-1}$, so
  a factor 1000 in the total integrated luminosity.
  

\item Electroweak gauge boson ($W$ and $Z$) production
  in association with charm quarks, relevant
  to pin down the strange and charm content of the proton.

\item The transverse momentum of $W$ and $Z$ bosons,
  relevant to constrain the gluon and the antiquarks.
  
\item Prompt photon production, providing a handle
  on the gluon at intermediate values of $x$.
    
\end{itemize}  

For each of these processes, we need from the experimentalists information
about the expected kinematical reach of the high-luminosity LHC, as well
as information on the binning, kinematic cuts, and the expected statistical
and systematic uncertainties.

We might want to study different scenarios for the statistical
uncertainties, from a more optimistic scenario to
a more pessimistic scenario.

\subsection{Theoretical calculations}

Concerning theoretical calculations, we will
generate the central value pseudo-data for these using the
PDF4LHC15 sets.
%
We will use NLO QCD
theory input and include NLO EW + PI contributions where relevant.
%
Note that being an exercise based on pseudo-data, the use of NNLO $K$-factors
is not needed since it would not affect the conclusions.

In order to generate the {\tt applgrids}, we can either use {\tt MCFM} or
{\tt aMC@NLO}.
%
If grids are generated by different people, we should make sure they have been
obtained with a uniform settings, to ensure consistency.

\subsection{Roadmap}

One problem here is that applying either reweighting or profiling to PDF4LHC15 directly
would not work, due to its hybrid nature.
%
So a better strategy would be to apply individual reweighting/profiling to the three
global sets and then combine them a posteriori a la PDF4LHC, to ensure a statistically
robust interpretation of the resulting uncertainties.

We should start with a given process, check that the whole machinery is in place, and
from there on keep adding more processes.
%
We will check with the SM WG conveners what is available.
%
For example, we could start with top quark pair production, and see what happens
with the gluon there.


\FloatBarrier

\phantomsection
\addcontentsline{toc}{section}{References}
%\input{nnpdf31smallx}
\bibliography{HLLHCPDFs}


\end{document}

%%%%%%%%%%%%%%%%%%%%%%%%%%%%%%%%%%%%%%%%%%%%%%%%%%%
%%%%%%%%%%%%%%%%%%%%%%%%%%%%%%%%%%%%%%%%%%%%%%%%%%%
