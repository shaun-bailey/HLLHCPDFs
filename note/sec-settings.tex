\section{Settings}
\label{sec:settings}

In this section we present the settings of the exercise that
we are going to carry out.
%
First of all we discuss the pseudo-data that will be used as an input to
the PDF fit.
%
Then we explain how the corresponding theoretical predictions have been produced.

\subsection{Pseudo-data generation}

Some of the processes that we will consider in this exercise
are:

\begin{itemize}

\item Drell-Yan - low to high mass, central to forward,
  relevant for quark flavour separation.

\item   Differential top quark pair production,
  relevant for the large $x$ gluon.

\item Inclusive jet production,
  useful to constrain both quarks and gluons.

  Measurements of inclusive jet and dijet production from ATLAS at 13 TeV are
  already available~\cite{Aaboud:2017wsi}
  based on a small luminosity of $\mathcal{L}=3.2$ fb$^{-1}$.
  %
  We would need to extrapolate these measurements to $\mathcal{L}=3$ ab$^{-1}$, so
  a factor 1000 in the total integrated luminosity.
  

\item Electroweak gauge boson ($W$ and $Z$) production
  in association with charm quarks, relevant
  to pin down the strange and charm content of the proton.

\item The transverse momentum of $W$ and $Z$ bosons,
  relevant to constrain the gluon and the antiquarks.
  
\item Prompt photon production, providing a handle
  on the gluon at intermediate values of $x$.
    
\end{itemize}  

For each of these processes, we need from the experimentalists information
about the expected kinematical reach of the high-luminosity LHC, as well
as information on the binning, kinematic cuts, and the expected statistical
and systematic uncertainties.

We might want to study different scenarios for the statistical
uncertainties, from a more optimistic scenario to
a more pessimistic scenario.

\subsection{Theoretical calculations}

Concerning theoretical calculations, we will
generate the central value pseudo-data for these using the
PDF4LHC15 sets.
%
We will use NLO QCD
theory input and include NLO EW + PI contributions where relevant.
%
Note that being an exercise based on pseudo-data, the use of NNLO $K$-factors
is not needed since it would not affect the conclusions.

In order to generate the {\tt applgrids}, we can either use {\tt MCFM} or
{\tt aMC@NLO}.
%
If grids are generated by different people, we should make sure they have been
obtained with a uniform settings, to ensure consistency.

\subsection{Roadmap}

One problem here is that applying either reweighting or profiling to PDF4LHC15 directly
would not work, due to its hybrid nature.
%
So a better strategy would be to apply individual reweighting/profiling to the three
global sets and then combine them a posteriori a la PDF4LHC, to ensure a statistically
robust interpretation of the resulting uncertainties.

We should start with a given process, check that the whole machinery is in place, and
from there on keep adding more processes.
%
We will check with the SM WG conveners what is available.
%
For example, we could start with top quark pair production, and see what happens
with the gluon there.
